%% Fix Bayes example numbers 
%% Fix KNN with different Ks figures 
%% 
%% 




\documentclass[12pt]{beamer}
\usepackage{xcolor}
\usepackage{pgf,pgfarrows,pgfnodes,pgfautomata,pgfheaps,pgfshade}
\usetheme{Air}

\DeclareMathOperator*{\argmax}{arg\,max}

\title[CSC349A Numerical Analysis]{CSC349A Numerical Analysis}


\logo{\pgfputat{\pgfxy(-0.5,7.5)}{\pgfbox[center,base]{\includegraphics[width=1.0cm]{figures/uvic}}}}  
\beamertemplatenavigationsymbolsempty

    \defbeamertemplate{footline}{author and page number}{%
      \usebeamercolor[fg]{page number in head/foot}%
      \usebeamerfont{page number in head/foot}%
      \hspace{1em}\insertshortauthor\hfill%
      \insertpagenumber\,/\,\insertpresentationendpage\kern1em\vskip2pt%
    }
    \setbeamertemplate{footline}[author and page number]{}



\subtitle[Leture 3]{Lecture 3}
\date[2025]{2025}
\author[George Tzanetakis]{George Tzanetakis}
\institute[University of Victoria]{University of Victoria}
%\logo{\includegraphics[scale=.25]{unilogo.pdf}}
\begin{document}
\frame{\maketitle} % <-- generate frame with title


\AtBeginSection[]
{
\begin{frame}<beamer>[allowframebreaks]{Table of Contents}
\tableofcontents[currentsection,currentsubsection, 
    hideothersubsections, 
    sectionstyle=show/shaded,
]
\end{frame}
}



\section{Approximations and Measuring Errors} 

\begin{frame}{Types of error}
\begin{itemize} 
\item{{\bf Formulation or modeling error}  arises because of
    incomplete mathematical models. For example, the simplified
    parachutist model we described will never exactly predict the
    motion of a real parachutist.}

\item{{\bf Data uncertainty/inherent erorr} Noisy data due to inexact
    measurements or observation.}

\item{{\bf Truncation error (Chapter 4)} results from using inexact approximations
    instead of an exact mathematical procedure such as the errors
    between the Euler method numerical procedure for predicting
    velocity versus the analytical solution obtained through
    differential calculus.}

\item{{\bf Roundoff error (Chapter 3)} is due to the fixed, finite
    precision representations used to represent real/complex numbers
    in computers.}


\end{itemize} 
\end{frame}


\begin{frame}{Ways to measure error}
\begin{itemize} 
\item{{\bf Significant digits} The digits of a number that can be used with confidence}
\vspace{\baselineskip}
\item{{\bf Absolute error} Difference between actual value and approximation}
\vspace{\baselineskip}
\item{{\bf Relative error} Normalized absolute error}
\vspace{\baselineskip}
\item{{\bf True error} Error when actual value is known}
\vspace{\baselineskip}
\item{{\bf Approximate error} Error when actual value is unknown}

\end{itemize} 
\end{frame}

\begin{frame}{Significant Digits (or Figures)}
\begin{itemize}
\item{We use significant digits in two contexts in this course.}
\item{In your experience you relate this term to the number of digits that you are using in a calculation. This is akin to roundoff error here.}
\item{We also use this term to denote the number of digits that are correct in an approximation, as in truncation error. They are really one and the same.}
\item{We don't count lead zeroes on the left of the radix point nor end zeroes on the right.}
\end{itemize}
\end{frame}

\begin{frame}{Significant Digits Examples}

{\bf Example 1} - For each of the following, determine the number of significant digits.
\begin{enumerate}
\item[(a)]{Let $p=13.685312$ and $p^*=13.69$.}
\vspace{0.5 in}
\item[(b)]{Let $p=1234321$ and $p^*=1234000$.} 
\vspace{0.5 in}
\item[(c)]{Let $p=0.001234$ and $p^*=0.001111$.}
\vspace{0.5 in}
\end{enumerate}
\end{frame}


\begin{frame}{Absolute Error}

If $p$ denotes the true (exact) value of some quantity, and $p^{*}$
denotes some approximation to $p$ then the {\bf absolute true error}
is defined as:

\begin{equation} 
|E_{t}| = | p - p ^{*}|
\end{equation} 

The absolute error only makes sense if you have a sense of the
magnitude of $p$, the quantity you are approximating.


\end{frame}

\begin{frame}{Absolute Error Examples}

{\bf Example 2} - For each of the following, determine the absolute error.
\begin{enumerate}
\item[(a)]{Let $p=13.685312$ and $p^*=13.69$.}
\vspace{0.5 in}
\item[(b)]{Let $p=1234321$ and $p^*=1234000$.} 
\vspace{0.5 in}
\item[(c)]{Let $p=0.001234$ and $p^*=0.001111$.}
\vspace{0.5 in}
\end{enumerate}
\end{frame}

\begin{frame}{Relative Error} 

A better approach is to introduce the concept of relative error, in
which the magnitude of the exact value $p$ is used to ``normalize''
the error. More specifically the {\bf relative error} is defined as: 

\begin{equation} 
|\varepsilon_{t}| = \frac{|p - p^{*}|}{|p|} = |1 - \frac{p^{*}}{p}| 
\end{equation} 

\end{frame} 

\begin{frame}{Relative Error Examples}

{\bf Example 3} - For each of the following, determine the relative error.
\begin{enumerate}
\item[(a)]{Let $p=13.685312$ and $p^*=13.69$.}
\vspace{0.5 in}
\item[(b)]{Let $p=1234321$ and $p^*=1234000$.} 
\vspace{0.5 in}
\item[(c)]{Let $p=0.001234$ and $p^*=0.001111$.}
\vspace{0.5 in}
\end{enumerate}
\end{frame}

\begin{frame}{Significant Digits vs. Relative Error} 
The {\it significant digits} is a discrete way of measuring error in
approximations in contrast to the relative error which is a continuous
way of measuring error in approximation. 
\end{frame}

\begin{frame}{Significant Digits vs. Relative Error Example } 

{\bf Example 4} - Let $p = \pi = 3.14159265 \dots$ and approximate it with $p_1=3.1$, $p_2=3.14$, etc., and compare relative error to significant digits for each.

\vspace{3 in}
\end{frame}

\begin{frame}{Significant Digits vs. Relative Error Example } 

{\bf Example 4} - Let $p = \pi = 3.14159265 \dots$ and approximate it with $p_1=3.1$, $p_2=3.14$, etc., and compare relative error to significant digits for each.

\vspace{0.5 in}

\begin{table} 
\begin{tabular}{|c|c|l|}
\hline 
approximations & number of significant digits & relative error
\\
\hline
3.1                   & 2       & 0.013 \\
\hline 
3.14                 & 3       & 0.00051 \\ 
\hline 
3.141               & 4       & 0.00019 \\ 
\hline 
3.1415             & 5       & 0.000029 \\ 
\hline 
\end{tabular} 
\end{table} 

\vspace{0.5 in}

\end{frame}

\begin{frame}{Relative error and significant digits} 

Frequently it is useful to relate the relative error to the number of
correct significant digits. This can be done with the
following: If

\begin{equation} 
|\varepsilon_{t} | < 5 \times 10 ^ {-n} 
\end{equation} 

\noindent 
then $p*$ approximates $p$ to {\bf $n$ significant digits}. 
\end{frame} 

\begin{frame}{When the true value is unknown} 
In order to calculate the relative or absolute error of a particular
quantity we need to know the value of the approximation $p^*$ as well
as the true value $p$. In many cases we don't know the true value. 
In fact numerical methods in engineering are used when we don't know
the true value otherwise we could just simply use it instead of
computing a numeric approximation. 

There are many algorithms in numerical methods that operate in an
iterative function in which an estimate of the solution is used to compute 
a better estimate and the process is repeated until some convergence
criterion is met. In this cases it is common to use our best current
estimate of the true value (i.e the previous estimate) to compute the
error compared to the current estimate. 
\end{frame}

\begin{frame}{Approximation to relative error} 

In this case of iterative algorithms we can use the following
approximation to the relative errror: 

\begin{equation} 
|\varepsilon_{a}| = \frac{|p_i - p_{i-1}|}{|p_{i}|} 
\end{equation} 

Similarly
we can use the following relation to conservatively estimate the number of
significant digits for the relative error in iterative algorithms: If

\begin{equation} 
|\varepsilon_{a}| < 0.5 \times 10^{-n}
\end{equation} 
then $p_i$ is correct to {\it at least} $n$ significant digits.
\end{frame} 

\begin{frame}{Approximate Relative Error Example} 

{\bf Example 5} - Mathematicians like to represent functions by infinite series. For
example the exponential function can be computed using: 

\begin{equation} 
e^{x} \approx 1 + x + \frac{x^2}{2} + \frac{x^3}{3!} + \dots + \frac{x^{i}}{i!}
\end{equation} 

\noindent 
As more terms are added to the series the approximation becomes better
and better. This is called a McLaurin series expansion. Let $p_{0}
=1$, $p_{1} = 1 + x$, $p_{2} = 1 + x + \frac{x^2}{2!}$ and so on. 

\vspace{2 in}

\end{frame} 

\begin{frame}{An example continued} 


\end{frame} 

\begin{frame}{Approximate at different precisions}
We can use this series approximation at different levels of precision 
to compute the value of $e^x$ for $x = 0.5$. We can use the true value 
$p = e^{0.5} = 1.648721 ..$ to compute the true relative error
$|\varepsilon_{t}|$. 


\begin{table} 
\begin{tabular}{|c|l|l|l|} 
\hline 
i & $p_{i}$ & $|\varepsilon_{t}| = \frac{|e^{0.5} - p_{i}|}{|e^{0.5}|}$ &
$|\varepsilon_{a}| = \frac{|p_{i} - p_{i-1}|}{|p_{i}|}$ \\
\hline 
0 & 1 & 0.393 & \\ 
1 & 1.5 & 0.0902 & 0.333 \\ 
2 & 1.625 & 0.0144 & 0.0769 \\ 
3 & 1.645833 & 0.00175 & 0.0127 \\ 
4 & 1.6484375 & 0.000172 & 0.00158 \\ 
5 & 1.6486979 & 0.0000142 & 0.000158 \\ 
\hline 


\end{tabular} 
\end{table} 

\end{frame}


\begin{frame} 
Note that the value of $|\varepsilon_t|$, which can only be computed if
we know the true (exact) answer $p$, can be used to estimate the
number of correct significant digits in each approximation. For
example: For $p_{2}$ we have $0.0144 < 5 \times 10^{-2}$ so the 
approximation $p_{2} = 1.625$ has at least 2 correct significant digits. For
$p_{4}$ we have $0.000172 < 5 \times 10^{-4}$ and therefore 
$p_{4} = 1.6484375$ has at least 4 correct significant digits. 
\end{frame}


\begin{frame}{What happens when true value is unknown ?} 
In practice, if a sequence of approximations to some unknown value 
is computed using an iterative algorithm, (we will use several such
algorithms during this course), then the exact relative error
$|\varepsilon_{t}|$ can not be computed. However, the relative error in
each approximation $p_i$ can be approximated by
$|\varepsilon_{a}|$ and can be used to estimate conservatively the
number of correct significant digits. For example for $i=5$ in the
table above we have $|\varepsilon_{a}| = 0.000158 < 0.5 x 10^{-3}$ 
implying that $p_{5} = 1.6486979$ has at least 3 correct significant
digits (notice that in fact it has 4 which means this is a
conservative estimate). 

\end{frame} 


\end{document}

