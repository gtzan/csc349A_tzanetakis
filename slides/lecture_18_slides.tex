\documentclass[12pt]{beamer}
\usepackage{xcolor}
\usepackage{pgf,pgfarrows,pgfnodes,pgfautomata,pgfheaps,pgfshade}
\usepackage{algorithm,algorithmic}
\usepackage{amsmath}
\usetheme{Air}

\DeclareMathOperator*{\argmax}{arg\,max}

\title[CSC349A Numerical Analysis]{CSC349A Numerical Analysis}


\logo{\pgfputat{\pgfxy(-0.5,7.5)}{\pgfbox[center,base]{\includegraphics[width=1.0cm]{figures/uvic}}}}  
\beamertemplatenavigationsymbolsempty

    \defbeamertemplate{footline}{author and page number}{%
      \usebeamercolor[fg]{page number in head/foot}%
      \usebeamerfont{page number in head/foot}%
      \hspace{1em}\insertshortauthor\hfill%
      \insertpagenumber\,/\,\insertpresentationendpage\kern1em\vskip2pt%
    }
    \setbeamertemplate{footline}[author and page number]{}



\subtitle[Lecture 18]{Lecture 18}
\date[2023]{2023}
\author[R. Little]{Rich Little}
\institute[University of Victoria]{University of Victoria}
%\logo{\includegraphics[scale=.25]{unilogo.pdf}}
\begin{document}
\frame{\maketitle} % <-- generate frame with title


\AtBeginSection[]
{
\begin{frame}<beamer>[allowframebreaks]{Table of Contents}
\tableofcontents[currentsection,currentsubsection, 
    hideothersubsections, 
    sectionstyle=show/shaded,
]
\end{frame}
}

\section{Numerical Differentiation Formulas}

\begin{frame}{Numerical Differentiation Formulas}
\begin{itemize}
\item{In Chapter 4, Taylor's Theorem is used to derive a numerical differentiation formula that was used to approximate a derivative in a mathematical model that was developed to determine the terminal velocity of a free-falling body (a parachutist).}

\item{Recall Taylor's Theorem (with $n=2$):
\[
f(x) = f(x_0) + f'(x_0)(x-x_0) + \frac{f''(x_0)}{2!}(x-x_0)^2 + \frac{f'''(\xi_1)}{3!}(x-x_0)^3
\]
where $\xi_1$ lies between $x$ and $x_0$.}
\end{itemize}

\end{frame}

\begin{frame}{Alternative Form of Taylor's Theorem}

Taylor's Theorem can also be stated in terms of a step size, $h<1$, by letting $h=x-x_0$ which implies that $x=x_0+h$. Thus, Taylor's approxiamtion becomes
\begin{equation}
f(x_0+h) = f(x_0) + f'(x_0)h + \frac{f''(x_0)}{2}h^2 + \frac{f'''(\xi_1)}{6}h^3
\end{equation}
where $\xi_1$ lies between $x_0$ and $x_0+h$.

\end{frame}

\begin{frame}{Forward Difference Approximation}

Solving for $f'(x_0)$ in (1) gives
\begin{align*}
f'(x_0)h =& f(x_0+h) - f(x_0) - \frac{f''(x_0)}{2}h^2 - \frac{f'''(\xi_1)}{6}h^3 \\
f'(x_0) =& \frac{f(x_0+h) - f(x_0)}{h}  - \frac{f''(x_0)}{2}h - \frac{f'''(\xi_1)}{6}h^2
\end{align*}
or
\[
f'(x_0) \approx \frac{f(x_0+h) - f(x_0)}{h} \text{ with error } O(h).
\]

\end{frame}

\begin{frame}{Another Form of Taylor's Theorem}

Consider yet another form of Taylor, where $x=x_0-h$, thus $-h=x-x_0$ and
\begin{equation}
f(x_0-h) = f(x_0) - f'(x_0)h + \frac{f''(x_0)}{2}h^2 - \frac{f'''(\xi_2)}{6}h^3
\end{equation}
where $\xi_2$ lies between $x_0-h$ and $x_0$.

\end{frame}

\begin{frame}{Backward Difference Approximation}

Solving for $f'(x_0)$ in (2) gives
\begin{align*}
f'(x_0)h =& f(x_0) - f(x_0-h) + \frac{f''(x_0)}{2}h^2 - \frac{f'''(\xi_2)}{6}h^3 \\
f'(x_0) =& \frac{f(x_0) - f(x_0-h)}{h} + \frac{f''(x_0)}{2}h - \frac{f'''(\xi_2)}{6}h^2
\end{align*}
or
\[
f'(x_0) \approx \frac{f(x_0) - f(x_0-h)}{h} \text{ with error } O(h).
\]


\end{frame}

\begin{frame}{Central Difference Approximation}
A more accurate approximation to a first derivative can be obtained by taking a linear combination of these two Taylor Theorem approximations.
Subtracting (2) from (1) gives
\begin{align*}
f(x_0+h)-f(x_0-h)&=\left[f(x_0) + f'(x_0)h + \frac{f''(x_0)}{2}h^2 + \frac{f'''(\xi_1)}{6}h^3\right] \\
&-\left[f(x_0) - f'(x_0)h + \frac{f''(x_0)}{2}h^2 - \frac{f'''(\xi_2)}{6}h^3\right]
\end{align*}
\vspace{1 in}
\end{frame}

\begin{frame}{Central Difference Approximation II}
\vspace{2 in}
\[
f'(x_0) \approx \frac{f(x_0+h) - f(x_0-h)}{2h} \text{ with error } O(h^2).
\]
\end{frame}

\begin{frame}{Example 1}
Use forward and central difference approximations to estimate the first derivative of
\[
f(x)=-0.1x^4-0.15x^3-0.5x^2-0.25x+1.25
\]
at $x=0.5$ using step size $h=0.5$.
\vspace{3 in}
\end{frame}

\begin{frame}{Example 1 continued}

\end{frame}

\section{High-Accuracy Differentiation Formulas}

\begin{frame}{High-Accuracy Differentiation Formulas}
\begin{itemize}
\item{Still more accurate approximations to $f'(x_0)$ can be obtained by taking linear combinations of more Taylor Theorem approximations and solving for $f'(x_0)$.}
\item{FIGURES 23.1 to 23.3 on pages 656-657 in the text summarize other such formulas.}
\end{itemize}
For example, Taylor Theorem approximations for each of
\[
f(x_0-2h),f(x_0-h),f(x_0+2h),f(x_0+h)
\]
can be used to derive the $O(h^4)$ approximation
\[
f'(x_0) \approx \frac{1}{12h} \left [f(x_0-2h)-8f(x_0-h)+8f(x_0+h)-f(x_0+2h) \right ]
\]
\end{frame}

\begin{frame}{Example 2}

\end{frame}

\begin{frame}{Example 2 continued}

\end{frame}

\section{Numerical Differentiation Formulas for Higher Derivatives}

\begin{frame}{Numerical Differentiation Formulas for Higher Derivatives}

Similar techniques can be applied to find formulas for higher derivatives. For example, we can derive $f''(x_0)$ by doing the following:

{\bf Example:} Let $n=3$ and use the following Taylor Approximations
\begin{equation*}
f(x_0+h) = f(x_0) + hf'(x_0) + \frac{h^2}{2}f''(x_0) + \frac{h^3}{6}f'''(x_0) + \frac{h^4}{24}f^{(4)}(\xi_1)
\end{equation*}
and
\begin{equation*}
f(x_0-h) = f(x_0) - hf'(x_0) + \frac{h^2}{2}f''(x_0) - \frac{h^3}{6}f'''(x_0) + \frac{h^4}{24}f^{(4)}(\xi_2)
\end{equation*}
\end{frame}

\begin{frame}{Numerical Differentiation Formula for $f''(x_0)$}

Adding these gives:
\begin{equation*}
f(x_0+h)+ f(x_0-h)= 2f(x_0) + h^2f''(x_0) +  \frac{h^4}{24}\left[f^{(4)}(\xi_1)+f^{(4)}(\xi_2)\right]
\end{equation*}
and solving for the desired derivative $f''(x_0)$ gives
\begin{equation*}
f''(x_0)= \frac{f(x_0+h)-2f(x_0)+f(x_0-h)}{h^2} - \frac{h^2}{24}\left[f^{(4)}(\xi_1)+f^{(4)}(\xi_2)\right]
\end{equation*}
That is, 
\begin{equation*}
f''(x_0) \approx \frac{f(x_0+h)-2f(x_0)+f(x_0-h)}{h^2}
\end{equation*}
with a truncation error of $O(h^2)$.

\end{frame}


\end{document} 



