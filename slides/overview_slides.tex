\documentclass[12pt]{beamer}
\usepackage{xcolor}
\usepackage{pgf,pgfarrows,pgfnodes,pgfautomata,pgfheaps,pgfshade}
\usepackage{algorithm,algorithmic}
\usetheme{Air}

\DeclareMathOperator*{\argmax}{arg\,max}

\title[CSC349A Numerical Analysis]{CSC349A Numerical Analysis}


\logo{\pgfputat{\pgfxy(-0.5,7.5)}{\pgfbox[center,base]{\includegraphics[width=1.0cm]{figures/uvic}}}}  
\beamertemplatenavigationsymbolsempty

    \defbeamertemplate{footline}{author and page number}{%
      \usebeamercolor[fg]{page number in head/foot}%
      \usebeamerfont{page number in head/foot}%
      \hspace{1em}\insertshortauthor\hfill%
      \insertpagenumber\,/\,\insertpresentationendpage\kern1em\vskip2pt%
    }
    \setbeamertemplate{footline}[author and page number]{}



\subtitle[Final Overview]{Final Overview}
\date[2021]{2021}
\author[R. Little]{Rich Little}
\institute[University of Victoria]{University of Victoria}
%\logo{\includegraphics[scale=.25]{unilogo.pdf}}
\begin{document}
\frame{\maketitle} % <-- generate frame with title


\AtBeginSection[]
{
\begin{frame}<beamer>[allowframebreaks]{Table of Contents}
\tableofcontents[currentsection,currentsubsection, 
    hideothersubsections, 
    sectionstyle=show/shaded,
]
\end{frame}
}


\begin{frame}{Final Logistics}
\begin{itemize} 

\item{The final exam is on Monday, December 13 from 2:00 to 5:00 p.m, McKinnon Gym Rows 23-32}

\item{The final is 3 hours long}

\item{The exam is closed book (see below regarding formula sheet)}

\item{Only simple, scientific calculators (the ones you use for math classes) are allowed. If you bring anything programmable or with a large screen and or internet access you will not be allowed to use it.}

\item{You can bring two single letter size (8.5 by 11) pieces of paper with formulas and notes (it can be double sided)}

\end{itemize} 

\end{frame} 

\begin{frame}{Final Material} 
\begin{itemize}
\item{The material covers the entire term (cumulative) }
\item{The first third material covered corresponds to part 1, chapters 1-4 of the textbook, lectures 1-7.} 
\item{The second third material covered corresponds to parts 2, 3 and 5, chapters 4-7, 9 and 18 of the textbook, lectures 8-17.} 
\item{The final third material covered corresponds to parts 6, and 7, chapters 21, 23, and 25 of the textbook.}
\item{There are 9 questions weighted towards the last third (2 from 1st, 2 from 2nd, 5 from 3rd)} 
\end{itemize}

{\bf Note:} This does NOT include MATLAB.

\end{frame} 

\section{Part 1 - Modeling, Computers, and Error Analysis}

\begin{frame}{Chapters 1 \& 2 - Mathematical Modeling and Programming}

\begin{itemize}

\item{Free-Falling Parachutist - Euler example}


\end{itemize}

\end{frame}

\begin{frame}{Chapter 3 - Approximations and Round-off Error}

\begin{itemize}

\item{Absolute vs. Relative Error vs. Significant digits} 

\item{True vs. Approximate Error}

\item{Number systems}

\item{Floating-point number system} 

\item{Idealized floating-point arithmetic} 

\item{Subtractive Cancellation}


\end{itemize}

\end{frame}

\begin{frame}{Chapter 4 - Truncation Errors and Taylor Series}

\begin{itemize}

\item{Truncation Error}

\item{Taylor's Theorem}

\item{Stability and Condition} 

\item{Condition number} 

\end{itemize}

\end{frame}

\section{Part 2 - Roots of Equations}

\begin{frame}{Chapter 5 - Bracketing Methods}

\begin{itemize}

\item{Bisection Method}

\item{Order of convergence of the Bisection method} 

\end{itemize}

\end{frame}

\begin{frame}{Chapter 6 - Open Methods}

\begin{itemize}

\item{Newton-Raphson Method}

\item{Newton method convergence}  

\item{Convergence Approximation}

\item{Quadratic convergence of Newton's method} 

\item{Secant method} 

\item{Order of convergence of the Secant method} 

\end{itemize}

\end{frame}

\begin{frame}{Chapter 7 - Roots of Polynomials}

\begin{itemize}

\item{Multiple Roots and the Multiplicity of a Zero} 

\item{Roots of Polynomials} 

\item{Horner's Algorithm (Nested Multiplication, Synthetic Division)} 

\item{Polynomial Deflation and Root polishing} 

\end{itemize}

\end{frame}

\section{Part 3 - Linear Algebraic Equations}

\begin{frame}{Chapter 9 - Gauss Elimination}

\begin{itemize}

\item{Systems of Linear Equations}

\item{Naive Gaussian Elimination}

\item{Counting floating point operations} 

\item{Gaussian Elimination with Partial Pivoting}

\item{Determinant of $A$}

\item{Gaussian Elimination with Scaling}


\end{itemize}

\end{frame}

\section{Part 5 - Curve Fitting}

\begin{frame}{Chapter 18 - Interpolation}

\begin{itemize}

\item{Polynomial interpolation} 

\item{Lagrange Interpolating Polynomial} 

\item{Error term of polynomial interpolation} 

\item{The Runge Phenomenon}

\item{Spline Interpolation} 

\item{Quadratic \& Cubic Spline Interpolants}

\end{itemize}

\end{frame}

\section{Part 6 - Numerical Differentiation and Integration}

\begin{frame}{Chapter 21 - Newton-Cotes Integration}

\begin{itemize}

\item{Numerical Integration (Quadrature)} 

\item{Newton-Cotes closed formulas} 

\item{Degree of precision} 

\item{Composite Newton-Cotes Formulas} 

\item{Newton-Cotes Formulas with Unequal Segments} 

\item{Open Newton-Cotes formulas} 

\end{itemize}

\end{frame}

\begin{frame}{Chapter 23 - Numerical Differentiation}

\begin{itemize}

\item{Forward, Backward and Central Finite-Divided Difference Formulas} 

\item{High-Accuracy Differentiation Formulas} 


\end{itemize}

\end{frame}

\section{Part 7 - Ordinary Differential Equations}

\begin{frame}{Chapter 25 - Runge-Kutta Methods}

\begin{itemize}

\item{Ordinary Differential Equations}

\item{Euler's Method}

\item{Geometric Interpretation of Euler's Method}

\item{Local vs. Global Error}

\item{Higher-Order Taylor Series Methods}

\item{Higher-Order Runge-Kutta Methods}

\end{itemize}

\end{frame}

\end{document} 



