\documentclass[12pt]{beamer}
\usepackage{xcolor}
\usepackage{pgf,pgfarrows,pgfnodes,pgfautomata,pgfheaps,pgfshade}
\usepackage{algorithm,algorithmic}
\usepackage{gensymb}
\usetheme{Air}

\DeclareMathOperator*{\argmax}{arg\,max}

\title[CSC349A Numerical Analysis]{CSC349A Numerical Analysis}


\logo{\pgfputat{\pgfxy(-0.5,7.5)}{\pgfbox[center,base]{\includegraphics[width=1.0cm]{figures/uvic}}}}  
\beamertemplatenavigationsymbolsempty

    \defbeamertemplate{footline}{author and page number}{%
      \usebeamercolor[fg]{page number in head/foot}%
      \usebeamerfont{page number in head/foot}%
      \hspace{1em}\insertshortauthor\hfill%
      \insertpagenumber\,/\,\insertpresentationendpage\kern1em\vskip2pt%
    }
    \setbeamertemplate{footline}[author and page number]{}



\subtitle[Lecture 14]{Lecture 14}
\date[2023]{2023}
\author[R. Little]{Rich Little}
\institute[University of Victoria]{University of Victoria}
%\logo{\includegraphics[scale=.25]{unilogo.pdf}}
\begin{document}
\frame{\maketitle} % <-- generate frame with title


\AtBeginSection[]
{
\begin{frame}<beamer>[allowframebreaks]{Table of Contents}
\tableofcontents[currentsection,currentsubsection, 
    hideothersubsections, 
    sectionstyle=show/shaded,
]
\end{frame}
}


\section{Approximation Theory/Curve Fitting} 


\begin{frame}{Introduction} 
Our next topic is the study of how a given function can be approximated by another function 
from a specified class of functions. The given function may be discrete or continuous. 

Typically the approximating function exhibits some desired properties such as: 

\begin{enumerate} 
\item{Continuity} 
\item{Easily differentiated} 
\item{Easily integrated} 
\item{Easily evaluated} 
\end{enumerate}
\end{frame}

\begin{frame}{Introduction II} 
Common classes of approximating functions: 
\begin{enumerate} 
\item{Polynomials} 
\item{Piecewise polynomials (splines)}
\item{Trigonometric sums (fourier series)}
\end{enumerate} 

We will also study criteria for what constitutes a ``good'' approximating function. 
\end{frame} 

\section{Polynomial interpolation} 

\begin{frame}{Polynomial interpolation} 

Recall that the general formula for an $n$th-order polynomial is
\[
P(x) = a_0 + a_1x + \dotsm + a_nx^n
\]
\noindent 
For $n+1$ distinct data points there is one and only one order $n$ (or less) polynomial that passes through them all. That is,
\begin{itemize}
\item{only one line that passes through two points}
\item{only one parabola that passes through three points, etc}
\end{itemize}
\vspace{2 in}
\end{frame} 

\begin{frame}{Polynomial interpolation II} 

{\bf Polynomial Interpolation} consists of determining the unique $n$th-order polynomial that fits the $n+1$ data points in question.
\begin{itemize}
\item{Although the polynomial is unique there are different methods for finding it and different formats for expressing it.}
\end{itemize}
\vspace{3 in}
\end{frame} 

\begin{frame}{Polynomial interpolation III} 

{\bf Formally:}
Let $y=f(x)$ be any given function. For any value of $n \geq 0$ and any given values 
$x_0, x_1, \dots, x_n$, let $y_i = f(x_i)$. 

\noindent
The {\bf polynomial interpolation problem} 
is to determine a polynomial $P(x)$ of degree less than or equal to $n$ for which: 
\[
P(x_i) = y_i \;\;\; \mbox{for } i=0,1,\dots,n 
\]

\begin{itemize}
\item{The set of $n+1$ data point $(x_i,y_i)$ may be the only functional values known (that is, $f(x)$ is 
a {\bf discrete function}, which could occur for example with experimental data), or} 
\item{$f(x)$ maybe be a known {\bf continuous function}, and the $n+1$ data points $(x_i, y_i)$ are a finite set of values with $y_i = f(x_i)$ (samples).}
\end{itemize} 
\end{frame} 

\begin{frame}{Terminology}
\begin{itemize} 
\item{If $z$ is some value between 2 of the given values $x_i$ and if $P(z)$ is computed as an approximation to $f(z)$, then this approximation is said to be determined by polynomial {\bf interpolation}.} 
\vspace{0.5 in}
\item{On the other hand, if $z$ lies outside of the interval containing all of the values $x_i$ and if $P(z)$ is computed as an approximation to $f(z)$, then this approximation is said to be determined by polynomial {\bf extrapolation}.}
\end{itemize}
\vspace{1 in}
\end{frame}

\begin{frame}{Polynomial interpolation vs. Taylor approximation} 
\begin{itemize}
\item{An {\bf interpolating polynomial} and the {\bf Taylor polynomial} both determine polynomial approximations to $f(x)$. However, in general they are very different approximations to $f(x)$.} 
\item{An interpolating polynomial uses the information: 

\[
y_0 = f(x_0), y_1 = f(x_1), \dots, y_n = f(x_n)
\]} 
\item{A Taylor polynomial uses the information: 
\[
f(x_0), f'(x_0), \dots, f^{(n)}(x_0)
\]}
\end{itemize}
\vspace{1 in}
\end{frame} 

\section{Lagrange Interpolating Polynomial} 

\begin{frame}{Lagrange Interpolating Polynomial}
Given ${(x_i,f(x_i))}, 0 \leq i \leq n$, with all $x_i$ distinct, consider the function: 
\begin{align*}
P(x) &= \sum_{i=0}^{n} L_i(x) f(x_i) \\
     &= L_0(x) f(x_0) + L_1(x)f(x_1) + \dots + L_n(x)f(x_n)
\end{align*} 
\noindent 
where 
\begin{align*}
L_i(x) &= \frac{(x-x_0)(x-x_1)\dots(x-x_{i-1})(x-x_{i+1})\dots(x-x_n)}{(x_i-x_0)(x_i-x_1)\dots(x_i-x_{i-1})(x_i-x_{i+1})\dots(x_i-x_n)} \\
   &= \prod_{j=0,j \neq i}^{n} \frac{x-x_j}{x_i-x_j}, \;\;\; \mbox{for } i=0,1,2,\dots,n 
\end{align*}
\end{frame}

\begin{frame}{Example 1}
Derive the general, order $n=1$, Lagrange interpolating polynomial.
\vspace{3 in}
\end{frame}

\begin{frame}{Example 2}
Derive the general, order $n=2$, Lagrange interpolating polynomial.
\vspace{3 in}
\end{frame}

\begin{frame}{Example 2 continued}
\end{frame}

\begin{frame}{Lagrange Interpolating Polynomial II}
Since each function $L_i(x)$ is a polynomial of order $n$ and $f(x_i)$ is a constant, $P(x)$ is a polynomial of order $\leq n$. 

Also, since 
\[
L_i(x_i) = 1 \mbox{ and } L_i(x_j) = 0 \mbox{ if } j \neq i,
\]
\noindent 
it follows that: 
\[
P(x_i) = f(x_i), \;\;\; \mbox{ for } i=0,1,2,\dots, n
\]
\noindent 
that is, $P(x)$ is an interpolating polynomial for the given data. It is called the {\bf Lagrange interpolating polynomial}. 
\end{frame}


\begin{frame}{Example 3}
Evaluate $\ln(2)$ using Lagrange polynomial interpolation, given that
\begin{align*}
\ln{1} &= 0 \\
\ln 4 &= 1.386294 \\
\ln 6 &= 1.791760
\end{align*}
\vspace{2 in}
\end{frame}

\begin{frame}{Example 3 continued}
\end{frame}

\begin{frame}{Example 4}
A complete elliptic integral function of the first kind is defined by
\[
K(k) = \int_{0}^{\pi/2} \frac{dz}{\sqrt{1 - k^2\sin^2{z}}}
\]
Interpolate $K(\sin{65.5 \degree})$ where
\begin{center}
\begin{tabular}{c|c} 
$\sin^{-1}{k}$ & $K(k)$ \\
 \hline
 $65\degree$ & $2.3088$ \\ 
 $66\degree$ & $2.3439$ \\ 
 $67\degree$ & $2.3809$ \\ 
\end{tabular}
\end{center}
\vspace{2 in}
\end{frame} 

\begin{frame}{Example 4 continued}
\end{frame}
\end{document} 



