\documentclass[12pt]{beamer}
\usepackage{xcolor}
\usepackage{pgf,pgfarrows,pgfnodes,pgfautomata,pgfheaps,pgfshade}
\usepackage{algorithm,algorithmic}
\usetheme{Air}

\DeclareMathOperator*{\argmax}{arg\,max}

\title[CSC349A Numerical Analysis]{CSC349A Numerical Analysis}


\logo{\pgfputat{\pgfxy(-0.5,7.5)}{\pgfbox[center,base]{\includegraphics[width=1.0cm]{figures/uvic}}}}  
\beamertemplatenavigationsymbolsempty

    \defbeamertemplate{footline}{author and page number}{%
      \usebeamercolor[fg]{page number in head/foot}%
      \usebeamerfont{page number in head/foot}%
      \hspace{1em}\insertshortauthor\hfill%
      \insertpagenumber\,/\,\insertpresentationendpage\kern1em\vskip2pt%
    }
    \setbeamertemplate{footline}[author and page number]{}



\subtitle[Lecture 20]{Lecture 20}
\date[2023]{2023}
\author[R. Little]{Rich Little}
\institute[University of Victoria]{University of Victoria}
%\logo{\includegraphics[scale=.25]{unilogo.pdf}}
\begin{document}
\frame{\maketitle} % <-- generate frame with title


\AtBeginSection[]
{
\begin{frame}<beamer>[allowframebreaks]{Table of Contents}
\tableofcontents[currentsection,currentsubsection, 
    hideothersubsections, 
    sectionstyle=show/shaded,
]
\end{frame}
}

\section{Higher-Order Taylor Series Methods}

\begin{frame}{Higher-Order Taylor Series Methods}

Higher order methods can be obtained by keeping more terms from the Taylor expansion.
\begin{multline*}
y(x_{i+1})=y(x_i)+hy'(x_i)+\frac{h^2}{2}y''(x_i)+\dotsm \\
	+\frac{h^n}{n!}y^{(n)}(x_i)+\frac{h^{n+1}}{(n+1)!}y^{(n+1)}(\xi_i) \\
=y(x_i)+hf(x_i,y(x_i))+\frac{h^2}{2}f'(x_i,y(x_i))+\dotsm \\
	+\frac{h^n}{n!}f^{(n-1)}(x_i,y(x_i))+O(h^{n+1})
\end{multline*}

\end{frame}

\begin{frame}{The Taylor Method of Order $n$}

Dropping the $O(h^{n+1})$ remainder term in the above Taylor expansion, gives a numerical method
\begin{multline*}
y_{i+1}=y_i+hf(x_i,y_i)+\frac{h^2}{2}f'(x_i,y_i)+\dotsm 
	+\frac{h^n}{n!}f^{(n-1)}(x_i,y_i)
\end{multline*}
for any integer $n \geq 1$.

This is called {\bf the Taylor method of order $n$} (as its local truncation error is $O(h^{n+1})$, and thus its global truncation error is $O(h^n)$).

Euler’s method is just the case when $n=1$.
\end{frame}

\begin{frame}{Example 1 - Taylor Method of Order 2}

Solve the differential equation $y'=y-x^2+1$ with $y(0)=0.5$ and step size $h=0.2$ using the Taylor Method of order $n=2$.
\[
y_{i+1}=y_i + hf(x_i,y_i) + \frac{h^2}{2}f'(x_i,y_i)
\]
\vspace{3 in}
\end{frame}

\begin{frame}{Example 1 continued}

\end{frame}

\section{Runge-Kutta Methods}

\begin{frame}{Runge-Kutta Methods} 

Advantage of Taylor methods of order $n$
\begin{itemize}
\item{global truncation error of $O(h^n)$ insures high accuracy (even for $n = 3, 4$ or $5$)}
\end{itemize}
Disadvantage
\begin{itemize}
\item{high order derivatives of $f(x,y(x))$ may be difficult and expensive to evaluate.}
\end{itemize}
\vspace{\baselineskip}
{\bf Runge-Kutta methods} are higher order formulas (they can have any order $\geq 1$) that require function evaluations only of  $f(x,y(x))$, and not of any of its derivatives.

\end{frame} 

\begin{frame}{General Form of RK Methods} 

Runge-Kutta methods are so-called one-step methods (as also are Euler’s method and all Taylor methods): that is, they are of the form
\[
y_{i+1}=y_i+h\Phi(x_i,y_i,h)
\]
for some (possibly very complicated) function $\Phi$.  

That is, each computed approximation $y_{i+1}$ is computed using only the value $y_i$ at the previous grid point, along with the values of $x_i$, the step size $h$, and of course the function $f(x,y(x))$ that specifies the differential equation.

\end{frame} 

\begin{frame}{General Form of RK Methods of Order $m$} 

A Runge-Kutta method of order $m$ is of the form:
\[
y_{i+1}=y_i+h\sum_{j=1}^m a_jk_j
\]
where the $a_j$ are constants and the $k_j$ are functions of the form,
\begin{align*}
k_1&=f(x_i,y_i) \\
k_j&=f(x_i+\alpha_jh,y_i+h\sum_{l=1}^{j-1}\beta_{jl}k_l), \text{ for } 2 \leq j \leq m
\end{align*}
\end{frame} 

\begin{frame}{Example 2 - Derive the general forms for $m=1,2,3,$ and $4$.}
\end{frame}

\begin{frame}{Example 2 - Derive the general forms for $m=1,2,3,$ and $4$.}
\end{frame}

\begin{frame}{The Goal given General Form of Order $m$} 
Each of these formuals defines a whole class of Runge-Kutta methods of order $m$.

Our goal is to take the formula for any fixed value of $m \geq 1$, and determine values for the parameters:
\begin{itemize}
\item{$\{a_1 \}$ when $m=1$}
\item{$\{a_1,a_2,\alpha_2,\beta_{21} \}$ when $m=2$}
\item{$\{a_1,a_2,a_3,\alpha_2,\alpha_3,\beta_{21},\beta_{31},\beta_{32} \}$ when $m=3$}
\item{$\{a_1,a_2,a_3,a_4,\alpha_2,\alpha_3,\alpha_4,\beta_{21},\beta_{31},\beta_{32},\beta_{41},\beta_{42},\beta_{43} \}$ when $m=4$}
\end{itemize}
so that the resulting Runge-Kutta method has as high an order as possible (i.e., its local truncation error is as small as possible).

\end{frame} 

\begin{frame}{Deriving RK Methods of Order $m$} 

This is accomplished by choosing the unknown parameters $\{a_i\},\{\alpha_i\}$, and $\{\beta_{ij}\}$ so that the Runge-Kutta formula
\[
y_{i+1}=y_i+h\sum_{j=1}^m a_jk_j
\]
is identical to the Taylor series expansion
\[
y_{i+1}=y_i+hy'_i+\frac{h^2}{2}y''_i+\frac{h^3}{6}y'''_i+\dotsm
\]
to as many terms as possible.

\end{frame} 


\section{Derivation of Runge-Kutta Methods}

\begin{frame}{Derivation of First Order RK Method}  

{\bf Case  m = 1}
 
The only Runge-Kutta method of first order 
\[
y_{i+1}=y_i+ha_1f(x_i,y_i)
\]
is when $a_1=1$. That is, Euler’s method
\[
y_{i+1}=y_i+hf(x_i,y_i)
\]
For each value of $m \geq 2$, there are an infinite number of Runge-Kutta formulas, each one having local truncation error $O(h^{m+1})$ and thus global truncation error $O(h^m)$.


\end{frame} 

\begin{frame}{Taylor Polynomial with 2 Variables} 
The derivation of Runge-Kutta methods and an understanding of why they work requires the Taylor polynomial for a function of 2 variables, but this Taylor polynomial is not required to use these methods to numerically approximate the solution of a differential equation.
\begin{multline*}
f(x+h,y+k)=f(x,y)+hf_x(x,y)+kf_y(x,y) \\
	+\frac{h^2}{2}f_{xx}(x,y) + hkf_{xy}(x,y)+\frac{k^2}{2}f_{yy}(x,y) \\
	+\frac{h^3}{6}f_{xxx}(x,y)+\frac{h^2k}{2}f_{xxy}(x,y)+\frac{hk^2}{2}f_{xyy}(x,y)+\frac{k^3}{6}f_{yyy}(x,y) \\
	+ \dotsm
\end{multline*}
where $f_x \equiv \frac{\partial f}{\partial x}, f_{xy} \equiv \frac{\partial^2 f}{\partial x \partial y}$, etc.

\end{frame} 

\begin{frame}{Derivation of second order R-K} 

{\bf Case  m = 2}
\begin{align*} 
y_{i+1} &=  y_i + a_1 h f(x_i, y_i)  \\
   &+ a_2 h f(x_i + \alpha_2 h, y_i + \beta_{21}hf(x_i,y_i))  \\ 
\end{align*} 
Using the Taylor exapnsion for $f(x_i + \alpha_2 h, y_i + \beta_{21}hf(x_i,y_i))$, we get
\begin{multline*} 
y_{i+1} = y_i + a_1 h f(x_i, y_i)+ a_2 h [ f(x_i,y_i) +   \alpha_2hf_x(x_i,y_i) \\
+ \beta_{21} h f (x_i, y_i) f_y(x_i,y_i) + O(h^2) ] \\
= y_i + [a_1 + a_2] h f(x_i, y_i) + h^2 [ a_2 \alpha_2 f_x(x_i,y_i) \\
+ a_2 \beta_{21}f(x_i,y_i)f_y(x_i,y_i)] + O(h^3) 
\end{multline*} 


\end{frame} 

\begin{frame}{Derivation of Second Order RK Method}  

But, also by Taylors Theorem 
\begin{align*} 
y_{i+1} =& y_i + h y'_i + \frac{h^2}{2} y''_i + O(h^3) \\
	=& y_i + hf(x_i,y_i) + \frac{h^2}{2} f'(x_i,y_i) + O(h^3) \\
           =& y_i + h f(x_i, y_i) + \frac{h^2}{2}[f_x(x_i,y_i) + f(x_i,y_i) f_y(x_i,y_i)] \\ 
	+& O(h^3) 
\end{align*} 

\end{frame} 

\begin{frame}{Derivation of Second Order RK Method}  
In summary, the second order Runge-Kutta general form is
\begin{align*} 
y_{i+1} = y_i + [a_1 + a_2] h f(x_i, y_i) + h^2 [ a_2 \alpha_2 f_x(x_i,y_i) \\
+ a_2 \beta_{21}f(x_i,y_i)f_y(x_i,y_i)]
\end{align*}
and the second order Taylor expansion is
\begin{align*} 
y_{i+1} =& y_i + h f(x_i, y_i) + h^2[\frac{1}{2}f_x(x_i,y_i) + \frac{1}{2}f(x_i,y_i) f_y(x_i,y_i)]
\end{align*}
These two are equal only when \[a_1+a_2=1, a_2 \alpha_2=1/2,a_2 \beta_{21}=1/2\]
\end{frame} 

\begin{frame}{Examples - Heun's Method} 

Let $a_1=a_2=1/2,\alpha_2=\beta_{21}=1$, which gives
\[
y_{i+1}=y_i+\frac{h}{2}\left[f(x_i,y_i)+f(x_i+h,y_i+hf(x_i,y_i))\right]
\]
\vspace{3 in}
\end{frame} 

\begin{frame}{Examples - Midpoint Method} 
Let $a_1=0,a_2=1,\alpha_2=\beta_{21}=1/2$, which gives
\[
y_{i+1}=y_i+hf(x_i+\frac{h}{2},y_i+\frac{h}{2}f(x_i,y_i))
\]
\vspace{3 in}
\end{frame} 

\begin{frame}{Example 2} 
Use Heun's Method to solve $y'=4e^{0.8x}-0.5y$ from $x=0$ to $x=4$ with $h=1$ and $y_0=2$.
\vspace{3 in}
\end{frame} 

\begin{frame}{Example 2 - Iterative refinement} 
We can make this an iterative process by plugging our current approximation of $y_{i+1}$ into Heun's for the Euler approximation.
\vspace{3 in}
\end{frame} 

\begin{frame}{Derivation of Third Order RK Methods}  

{\bf Case  m = 3}
It can be shown that any solution of a certain system of 6 nonlinear equations in 8 unknowns gives a third-order Runge-Kutta Method.

One common solution is
\[
a_1=\frac{1}{6}, a_2=\frac{2}{3}, a_3=\frac{1}{6}, \alpha_2=\frac{1}{2}, \alpha_3=1, \beta_{21}=\frac{1}{2}, \beta_{31}=-1, \beta_{32}=2
\]
which gives the third-order Runge-Kutta method
\[
y_{i+1}=y_i+\frac{h}{6}(k_1+4k_2+k_3)
\]
where
\begin{align*}
k_1 &= f(x_i,y_i) \\
k_2 & = f\left(x_i+\frac{h}{2},y_i+\frac{h}{2}k_1\right) \\
k_3 &= f(x_i+h,y_i-hk_1+2hk_2)
\end{align*}
\end{frame} 

\begin{frame}{Derivation of Fourth Order RK Methods}  

{\bf Case  m = 4}
The 13 Runge-Kutta parameters are obtained by solving 11 nonlinear equations in 13 unknowns. One solution is called the {\bf "classical" Runge-Kutta method}, which has global truncation error of $O(h^4)$:

\[
y_{i+1}=y_i+\frac{h}{6}(k_1+2_2+2k_3+k_4)
\]
where
\begin{align*}
k_1 &= f(x_i,y_i) \\
k_2 & = f\left(x_i+\frac{h}{2},y_i+\frac{h}{2}k_1\right) \\
k_3 & = f\left(x_i+\frac{h}{2},y_i+\frac{h}{2}k_2\right) \\
k_4 &= f(x_i+h,y_i+hk_3)
\end{align*}
\end{frame} 

\begin{frame}{Classical Fourth Order RK Method Graphically}  

\end{frame} 

\begin{frame}{Example 3} 
Use the classical 4th-order RK method with $h=0.2$ and $y(0)=0.5$ to solve $y'=y-x^2+1$.
\vspace{3 in}
\end{frame} 

\begin{frame}{Example 3 continued} 

\end{frame} 

\end{document} 



