\documentclass[12pt]{beamer}
\usepackage{xcolor}
\usepackage{pgf,pgfarrows,pgfnodes,pgfautomata,pgfheaps,pgfshade}
\usetheme{Air}

\DeclareMathOperator*{\argmax}{arg\,max}

\title[CSC349A Numerical Analysis]{CSC349A Numerical Analysis}


\logo{\pgfputat{\pgfxy(-0.5,7.5)}{\pgfbox[center,base]{\includegraphics[width=1.0cm]{figures/uvic}}}}  
\beamertemplatenavigationsymbolsempty

    \defbeamertemplate{footline}{author and page number}{%
      \usebeamercolor[fg]{page number in head/foot}%
      \usebeamerfont{page number in head/foot}%
      \hspace{1em}\insertshortauthor\hfill%
      \insertpagenumber\,/\,\insertpresentationendpage\kern1em\vskip2pt%
    }
    \setbeamertemplate{footline}[author and page number]{}



\subtitle[Lecture 6]{Lecture 6}
\date[2023]{2023}
\author[R. Little]{Rich Little}
\institute[University of Victoria]{University of Victoria}
%\logo{\includegraphics[scale=.25]{unilogo.pdf}}
\begin{document}
\frame{\maketitle} % <-- generate frame with title


\AtBeginSection[]
{
\begin{frame}<beamer>[allowframebreaks]{Table of Contents}
\tableofcontents[currentsection,currentsubsection, 
    hideothersubsections, 
    sectionstyle=show/shaded,
]
\end{frame}
}

\section{Taylor's theorem} 

\begin{frame}{Truncation Errors} 

{\bf Truncation errors} occur when some exact mathematical procedure is
replaced by a finite approximation. Examples: 
\begin{itemize} 
\item Approximation of a function by a finite number of terms: 
\[ 
e^{x} \approx 1 + x + \frac{x^2}{2} + \frac{x^3}{6}
\]
\item Approximation of a derivative by a finite difference: 
\[
\frac{dv}{dt}|_{t=t_i} \approx \frac{v(t_{i+1}) - v(t_{i})}{t_{i+1} - t_{i}}
\]
\item Approximation of a definite integral by a finite sum:
\[
\int_{a}^{b} f(x)dx \approx \frac{b-a}{2} \left[ f(a) + f(b) \right]
\]
\end{itemize} 


\end{frame} 




\begin{frame}{Taylor's theorem}

{\bf Taylor's Theorem} is the fundamental tool for deriving and analyzing numerical
approximation formulas in this course. 

\begin{itemize}

\item{It states that any ``smooth'' function (one with a sufficient number of derivatives) can be
approximated by a polynomial, and it includes an error (remainder)
term that indicates how accurate the polynomial approximation
is.}

\item{Taylor's theorem also provides a means to estimate the value of a
function $f(x)$ at some point $x_{i+1}$ using the values of $f(x)$ and
its derivatives at some nearby point $x_{i}$.}

\end{itemize}
\end{frame} 

\begin{frame}{Taylor's theorem}

\begin{theorem}
Let $n \geq 0$ and let $a$ be any constant. If $f(x)$ and its first $n+1$ derivatives are continuous on interval containing $x$ and $a$, then: 

\begin{eqnarray*}
 f(x) = f(a) + f'(a)(x-a) + \frac{f''(a)}{2!}(x-a)^2 + \\ \frac{f'''(a)}{3!}(x-a)^3 + \dots + \frac{f^{n}(a)}{n!}(x-a)^{n} + R_n
\end{eqnarray*}
\noindent 
where $R_n$ is the {\it remainder} term.

\end{theorem}



\end{frame} 

\begin{frame}{Taylor polynomial} 
\begin{eqnarray*}
P_n(x) = f(a) + f'(a)(x-a) + \frac{f''(a)}{2!}(x-a)^2 + \\ \frac{f'''(a)}{3!}(x-a)^3 + \dots + \frac{f^{n}(a)}{n!}(x-a)^{n} 
\end{eqnarray*}
\noindent 
is a polynomial of degree $n$ in $x$, and is called the {\bf Taylor polynomial approximation of degree $n$} for f(x) expanded about $a$.  
\end{frame}

\begin{frame}{Truncation error}

 The remainder term, $R_{n}$, is the {\bf truncation error} of the Taylor polynomial approximation to $f(x)$.

\begin{eqnarray*}
R_{n} = \frac{f^{(n+1)}(\xi)}{(n+1)!} (x-a)^{n+1} 
\end{eqnarray*}
\noindent 
where $\xi$ is some value between $x$ and $a$. 


\end{frame} 

\section{Taylor Examples} 

\begin{frame}{Example 1}
(a) Determine the Taylor polynomial approximation of order $n=3$ for 
$f(x) = \ln(x+1)$ expanded about $a=0$.
\vspace{3 in}
\end{frame}

\begin{frame}{Example 1 continued}
(b) Using (a) approximate $\ln(1.25)$.
\vspace{3 in}
\end{frame}

\begin{frame}{Example 1 continued}
(c) Determine a good upper bound on the error of the approximation in (b). 
\vspace{3 in}
\end{frame}

\begin{frame}{Truncation error on interval} 
Rather than just using the Taylor polynomial approximation to estimate the value of a function at one specified point, it is more common to use the polynomial approximation {\bf for an entire interval of values x}. In such a case, it is also desirable to be able to determine the accuracy (that is an upper bound for the error). 
\end{frame} 

\begin{frame}{Example 1 continued}
(d) Determine a good upper bound on the error of approximating $\ln(x+1)$ for any $x$ in the interval $[0,0.5]$. 
\vspace{3 in}
\end{frame}

\begin{frame}{The remainder term}
Consider the remainder term when $n=0$, expanded about some $a$. Then, $f(x) \approx f(a)$ with error term $R_0=f'(\xi)(x-a)$ for some $\xi$ between $x$ and $a$.
\vspace{3 in}
\end{frame}

\begin{frame}{The Mean-Value Theorem}
\begin{definition}
Let $f$ be continuous on interval $[a,b]$ and differentiable on $(a,b)$. Then, there exists a $c \in (a,b)$ such that

\[
f'(c)=\frac{f(b)-f(a)}{b-a}
\]
\end{definition}
\end{frame}

\begin{frame}{Take another look at Taylor} 
What are we doing here?
\begin{itemize}
\item{We know $f(a)$, $f'(a)$,$f''(a)$, etc.}
\item{We approxiamte some neighbouring $f(x)$.}
\item{That is, if we know the value of a function and its derivatives at some point, we can predict its value at some other (nearby) point.}
\item{What does this describe?}
\end{itemize}
\end{frame} 

\begin{frame}{An iterative process for approximating a function} 
Let $x_{i+1} = x_{i} + h$ so that $h = x_{i+1} - x_{i}$. Then Taylor's theorem for $f(x)$ expanded about $x_{i}$, and evaluated at $x=x_{i+1}$ is: 

\[ 
f(x_{i+1}) = f(x_i) + f'(x_i) h + \frac{f''(x_i)}{2!}h^2 + \frac{f'''(x_i)}{3!}h^3 + \dots + \frac{f^{n}(x_i)}{n!}h^n + R_{n}  
\] 

\end{frame} 

\begin{frame}{Example 2}
Use the alternative form of Taylor's theorem, with $n=1$, to derive an approximation of $f'(x_i)$ with an associated error term.
\vspace{3 in}
\end{frame}

\begin{frame}{Approximating first derivative}
This gives the first derivative approximation
\[
f'(x_i) \approx \frac{f(x_{i+1})-f(x_i)}{h}
\] 
\noindent 
that was used in Chapter 1; it also gives the trunaction error of this finite difference approximation to the derivative, namely $-\frac{R_1}{h} = -\frac{f''(\xi)}{2}h$. As this is some constant times $h$, we say that this truncation error is $O(h)$. 

\end{frame} 


\begin{frame}{Example 3} 

The Taylor polynomial approximation for $f(x) = e^x$ expanded about $a=0$ is 

\[ 
e^x \approx 1 + x + \frac{x^2}{2!} + \frac{x^3}{3!} + \dots + \frac{x^n}{n!} 
\]
\noindent 
With remainder term, 
\[
R_n = \frac{f^{(n+1)} (\xi)}{(n+1)!}(x-a)^{n+1} 
\]
\end{frame} 

\begin{frame}{Example 3} 
The truncation error of any Taylor polynomial approximation is small when $x$ is close to $a$ (note that $R_n=0$ when $x=a$) and will increase as $x$ gets further away from $a$. Also, as $n$ increases, the Taylor polynomial approximations become better and better approximations to $f(x)$, provided of course that $f^{(n+1)}(x)$ is bounded on some interval containing $x$ and $a$. 

\end{frame} 

\begin{frame}{Alternative form of remainder}
Alternative form for the remainder $R_n$ which can also be
defined as: 

\begin{equation} 
R_n = \int^{x}_{a} \frac{(x-t)^n}{n!} f^{(n+1)}(t)dt 
\end{equation} 
\end{frame} 

\begin{frame}{Derivation} 
The derivation is based on
the first theorem of mean for integrals which states that if a
function $f$ is continuous and integrable on an interval containing 
$\alpha$ and $x$, then there exists a point $\xi$ between $\alpha$
and $x$ such that: 

\begin{equation} 
\int_{\alpha}^{x} g(t)dt = g(\xi)(x-\alpha) 
\end{equation} 
\end{frame}


\end{document}

