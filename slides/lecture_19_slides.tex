\documentclass[12pt]{beamer}
\usepackage{xcolor}
\usepackage{pgf,pgfarrows,pgfnodes,pgfautomata,pgfheaps,pgfshade}
\usepackage{algorithm,algorithmic}
\usepackage{amsmath}
\usetheme{Air}

\DeclareMathOperator*{\argmax}{arg\,max}

\title[CSC349A Numerical Analysis]{CSC349A Numerical Analysis}


\logo{\pgfputat{\pgfxy(-0.5,7.5)}{\pgfbox[center,base]{\includegraphics[width=1.0cm]{figures/uvic}}}}  
\beamertemplatenavigationsymbolsempty

    \defbeamertemplate{footline}{author and page number}{%
      \usebeamercolor[fg]{page number in head/foot}%
      \usebeamerfont{page number in head/foot}%
      \hspace{1em}\insertshortauthor\hfill%
      \insertpagenumber\,/\,\insertpresentationendpage\kern1em\vskip2pt%
    }
    \setbeamertemplate{footline}[author and page number]{}



\subtitle[Lecture 19]{Lecture 19}
\date[2023]{2023}
\author[R. Little]{Rich Little}
\institute[University of Victoria]{University of Victoria}
%\logo{\includegraphics[scale=.25]{unilogo.pdf}}
\begin{document}
\frame{\maketitle} % <-- generate frame with title


\AtBeginSection[]
{
\begin{frame}<beamer>[allowframebreaks]{Table of Contents}
\tableofcontents[currentsection,currentsubsection, 
    hideothersubsections, 
    sectionstyle=show/shaded,
]
\end{frame}
}

\section{Ordinary Differential Equations}

\begin{frame}{Ordinary Differential Equations}

\begin{itemize}
\item{Corresponds to Handouts 34-37 and Chapter 25 in the text.}
\item{We did this way back in Lecture 1 when we solved the parachutist problem analytically and numerically.}
\end{itemize}

\end{frame}

\begin{frame}{Chapter 25: Runge-Kutta Methods}

\begin{itemize}
\item{These correspond to solving first-order, initial-value ordinary differential equations.}
\item{That is, solving for $y(x)$ given,
\[
y'(x)=f(x,y(x)) \text{ and } y(x_0) = y_0
\]}
\item{Here, $x$ is the independent variable, $y$ the dependent variable as a function of $x$, and $f(x,y(x))$ is a function of $x$ and $y$.}
\item{The goal is to determine $y(x)$ over $a \leq x \leq b$ given $a=x_0,b,y_0$ and $f(x,y(x))$.}
\end{itemize}

\end{frame}

\begin{frame}{Analytic Method}

Determine $y(x)$ for $0 \leq x \leq 2$ such that
\[
y'(x) = y - x^2 + 1
\]
subject to $y(0) = 0.5$.

\vspace{\baselineskip}
This problem has an analytic solution, namely \[y(x) = (x+1)^2 - 0.5e^x\].

\end{frame}

\begin{frame}{Numerical Methods}

All numerical methods we will consider for approximating $y(x)$ are called {\bf difference methods}:  that is, the continuous, exact solution $y(x)$ is approximated by a \underline{finite set} of computed values at a set of \underline{mesh points} $x_0,x_1,...,x_N$  in $[a,b]$.

For now, we consider only equally-spaced mesh points, and let
\[
x_i=a+ih=x_0+ih, \text{ for } i=0,1,2,...,N
\]
where
\[
h=\frac{b-a}{N}					 
\]
is called the \underline{step size}.

\end{frame}

\begin{frame}{Numerical Methods Graphically}

Notation: At each of the $N+1$ mesh points $x_i$, the exact solution is denoted by $y(x_i)$ and the approximate solution by $y_i$.

\begin{figure}[h] 
  \centering
  \includegraphics[scale=0.7]{solving_odes}
  \label{fig:solving_odes}
\end{figure}

\end{frame}

\begin{frame}{Euler's Method}

In general, we derive iterative formulas of the form
\[ y_{i+1}=y_i + \phi h \]
where $\phi$ is the slope of the tangent at $x_i$.
\begin{itemize}
\item{Usually, we approximate $\phi$. This is done in different ways for different methods.}
\item{In Euler's Method the right hand side of the differential equation itself gives us an approxiamtion of $\phi$ at each point.}
\item{Recall, $y'(x)=f(x,y(x))$ is the form of the ODEs we are looking at.}
\item{So, at each step $i$ we let $\phi = f(x_i,y(x_i))$.}
\end{itemize}

\end{frame}

\section{Euler's Method}


\begin{frame}{Derivation of Euler's Method}

Taylor's Theorem for $y(x)$ (with $n=1$) expanded about $x_i$ is
\[
y(x)=y(x_i)+hy'(x_i)+\frac{h^2}{2}y''(\xi)
 \]
for some value $\xi$ between $x_i$ and $x$.  Thus, at $x=x_{i+1}$,
\[
y(x_{i+1})=y(x_i)+hy'(x_i)+\frac{h^2}{2}y''(\xi)
\]
where $h=x_{i+1}-x_i$.

\end{frame}

\begin{frame}{Derivation of Euler's Method II}

For small $h$, this suggests the approximation
\[
y(x_{i+1}) \approx y(x_i)+hy'(x_i)
=y(x_i)+hf(x_i,y(x_i)),
 \]
using the differential equation $y'(x)=f(x,y(x))$. 

Here, we used exact soultions $y(x_i)$ but at each itereation we generally use previous approximations $y_i$. 

So, Euler's Method approxiamtes each $y(x_{i+1})$ by
\[
y_{i+1}=y_i+hf(x_i,y_i)
\]
for $i=0,1,2,...,N$ where $y_0=y(x_0)$ is the initial condition.

\end{frame}

\begin{frame}{Example 1: Euler's Method}

Consider the initial-value problem
\[
y'(x)=y-x^2+1, \text{ subject to } y(0)=0.5.
\]
The first few computed approximations $y_i$ and the corresponding exact solutions $y(x_i)$ are as follows:
\vspace{3 in}
\end{frame}

\begin{frame}{Example 1 continued}

\end{frame}

\begin{frame}{Example 1 Summary}

Consider the initial-value problem
\[
y'(x)=y-x^2+1, \text{ subject to } y(0)=0.5.
\]
The first few computed approximations $y_i$ and the corresponding exact solutions $y(x_i)$ are as follows:

\begin{center}
\begin{tabular}{c|c|c|c}
$x_i$ & $y_i$ & $y(x_i)$ & $|y(x_i)-y_i|$ \\
\hline
0 & 0.5 & 0.5 & 0 \\
0.2 & 0.8 & 0.8292986 & 0.0292986 \\
0.4 & 1.152 & 1.1240877 & 0.0620877 \\
0.6 & 1.5504 & 1.6489406 & 0.0985406 \\
0.8 & 1.98848 & 2.1272295 & 0.1387495
\end{tabular}
\end{center}

\end{frame}

\section{Geometric Interpretation of Euler's Method}

\begin{frame}{Geometric Interpretation of Euler's Method}
$y_0=y(x_0)$ is the intitial condition. Using this value, we compute
\[
y_1=y_0+hf(x_0,y_0)=y_0+hy'(x_0)
\]
from which it follows that
\[
y'(x_0)=\frac{y_1-y_0}{h}=\frac{y_1-y_0}{x_1-x_0}
\]
Geometrically, this says that $y_1$ is obtained from $y_0$ by constructing the tangent line to the graph of $y(x)$ at $x_0$ (which has slope equal to $y'(x_0)$) and going a distance $h$.

\end{frame}

\begin{frame}{Geometric Interpretation of Euler's Method}

\begin{figure}[h] 
  \centering
  \includegraphics[scale=0.65]{eulers}
  \label{fig:eulers}
\end{figure}

Similarly,
\[
y_2=y_1+hf(x_1,y_1)
\]
so $y_2$ is determined by constructing a straight line through $(x_1,y_1)$ with slope $f(x_1,y_1)$ (not $f(x_1,y(x_1))$, the exact slope).  

\end{frame}

\begin{frame}{Notes on Euler's Method}

In general, $y_{i+1}$ is obtained by constructing a straight line through $(x_i,y_i)$ with slope equal to $f(x_i,y_i)$, which is an approximation to $y'(x_i)$.

\vspace{\baselineskip}

As $y_i$ depends on $y_{i-1}$, which in turn depends on $y_{i-2}$ and so on, successive values of $y_i$ tend to be less and less accurate (as the truncation errors accumulate as you go across the interval $[a,b]$).

\end{frame}

\section{Error Analysis of Euler's Method}

\begin{frame}{Truncation Error}

The {\bf total truncation error} in each computed approximation $y_{i+1}$ is composed of two parts:
\begin{enumerate}
\item{the {\bf local truncation error} is the amount of truncation error that results from \underline{a single application} of a numerical method (that is, from the computation of $y_{i+1}$ from $y_i$), and}
\item{the {\bf global truncation error} contains the \underline{accumulated local truncation errors} from all of the steps leading up to the computation of $y_{i+1}$.}
\end{enumerate}

\end{frame}

\begin{frame}{Global Truncation Error}

{\bf Definition:} $|y(x_i)-y_i|$ is called the {\bf global truncation error} at $x_i$.

\vspace{\baselineskip}

{\bf Definition:} If the global truncation error is $O(h^k)$, the numerical method used to compute the values $y_i$ is said to be of {\bf order $k$} (or a $k^{th}$ order method).

\vspace{\baselineskip}

{\bf Definition} A numerical method is said to be {\bf convergent} (with respect to the differential equation it approximates) if
\[
\lim_{h\to0} \max_{1 \leq i \leq N} |y(x_i)-y_i|=0
\]
				 .
\end{frame}

\begin{frame}{Notes on Global Truncation Error}

The order of a method is a measure of the accuracy of the computed approximations, or of the rate of convergence of the computed approximations $y_i$ to the exact solutions $y(x_i)$ as $h \rightarrow 0$.

\vspace{\baselineskip}

For any fixed value of the step size $h$, the \underline{larger the order} $k$, the more accurate are the computed approximations.
			 .
\end{frame}

\begin{frame}{Local Truncation Error}

{\bf Definition:} The {\bf local truncation error} at any point $x_{i+1}$ is the amount of truncation error that would result from using a numerical method with the exact value $y(x_i)$ rather than the computed approximation $y_i$

\begin{figure}[h] 
  \centering
  \includegraphics[scale=0.65]{local_trunc}
  \label{fig:local_trunc}
\end{figure}

\end{frame}

\begin{frame}{Derivation of Local Truncation Error for Euler's Method}

Recall, Euler’s method is
\[
y_{i+1}=y_i+hf(x_i,y_i).
\]
Using the exact value $y(x_i)$ in this formula instead of the computed approximation $y_i$, define
\begin{equation}
v_{i+1}=y(x_i)+hf(x_i,y(x_i)).
\end{equation}
Then the local truncation error at $x_{i+1}$ is equal to
\[
|y(x_{i+1})-v_{i+1}|.
\]

\end{frame}

\begin{frame}{Example 3: Local Error}

Recall the initial-value problem $y'(x)=y-x^2+1$, subject to $y(0)=0.5$. The first two computed approximations are:

\begin{align*}
&y_1=0.8, y(x_1)=0.829298621, |E_t|=0.0292986 \\
&y_2=1.152, y(x_2)=1.2140877, |E_t|=0.0620877
\end{align*}

\vspace{2 in}

\end{frame}

\begin{frame}{Order of the Local Truncation Error for Euler's Method}

We use Taylor's Theorem,
\[
y(x_{i+1})=y(x_i)+hy'(x_i)+\frac{h^2}{2}y''(\xi_i)
\]
where $h=x_{i+1}-x_i$ and $\xi_i \in [x_i,x_{i+1}]$. So,
\begin{equation}
y(x_{i+1})=y(x_i)+hf(x_i,y(x_i))+\frac{h^2}{2}y''(\xi_i).
\end{equation}
Now, (2)-(1) gives,
\[
|y(x_{i+1})-v_{i+1}|=\left |\frac{h^2}{2}y''(\xi_i) \right |.
\]
Thus, Euler's Method is $O(h^2)$ locally.

\end{frame}

\begin{frame}{Relationship between Local and Global Error for Euler's Method}

An informal justification for the relationship between the local and global truncation errors for Euler's method is as follows:

The local truncation error in each step of Euler's method is, as shown above, $O(h^2)$.

After $N$ steps of Euler's method, the global truncation error $|y(x_N)-y_N|$ of the final computed approximation $y_N$ at $x_N$ will depend on the $N$ local truncation errors of $y_1,y_2,...,y_N$.

But the magnitude of $N$ local truncation errors is 
\[
N \times O(h^2)=\frac{b-a}{h} \times O(h^2) = O(h) \text{ since } h = \frac{b-a}{N}
\]

\end{frame}

\begin{frame}{Relationship between Local and Global Error in General}

{\bf Theorem:} (for any method)
If the local truncation error is $O(h^{k+1})$, then the global truncation error is $O(h^k)$. That is, the numerical method used to compute the approximate solution has {\bf order $k$}.  

Thus, the global truncation error for Euler’s method is $O(h)$, and Euler’s method has order 1.

Usually, you cannot compute either the global or local errors exactly. Also, the global error is difficult to approximate directly. But, there are good ways of approximating the local errors which can then be used to approximate the global error.

\end{frame}

\begin{frame}{Disadvantages of Euler's Method}
\begin{itemize}
\item{Euler's method is not sufficiently accurate.}
\item{Because the global truncation error is only $O(h)$, a very small step size $h$ is required to compute highly accurate approximations.}
\item{Euler is not often used.}
\item{Methods with a higher order of accuracy are used in practice.}
\end{itemize}
\end{frame}


\end{document} 



