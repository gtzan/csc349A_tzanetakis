\documentclass [titlepage,12pt,letter] {article}
\pagestyle{myheadings}


\usepackage{graphicx} 
\usepackage{epsfig}
\usepackage{subfigure}
\usepackage{fancyhdr}
\usepackage{url} 
\usepackage{amsmath}
\usepackage{algorithm} 
\usepackage{algorithmic}
\pagestyle{fancy}



\fancyhead{}
\fancyfoot{}
			
\lhead{CSC349A Lecture Notes}
\rhead{Little, Rich}


\setcounter{page}{1}
\cfoot{\thepage}




\begin{document} 


These are the lecture notes for CSC349A Numerical Analysis taught by
Rich Little. They roughly correspond to
the material covered in each lecture in the classroom but the actual
classroom presentation might deviate significantly from them depending
on the flow of the course delivery. They are provided as a reference to
the instructor as well as supporting material for students who miss
the lectures. They are simply notes to support the lecture so the text
is not detailed and they are not thoroughly checked. Use at your own
risk. They are complimentary to the handouts. Many thanks to all the
guidance and materials I received from Dale Olesky who has taught this
course for many years and George Tzanetakis. 

\section{Numerical Differentiation Formulas}

In Chapter 4, Taylor’s Theorem was used to derive a numerical differentiation formula that was used to approximate a derivative in a mathematical model that was developed to determine the terminal velocity of a free-falling body (a parachutist).

Recall Taylor's Theorem (with $n=2$):
\[
f(x) = f(x_0) + (x-x_0)f'(x_0) + \frac{(x-x_0)^2}{2!}f''(x_0) + \frac{(x-x_0)^3}{3!}f'''(\xi_1)
\]
where $\xi_1$ lies between $x$ and $x_0$ for some given $x_0$. We will use this formula to come up with our first approximation formula of the derivative, $f'(x_0)$, called the {\it Forward Finite-Divided Difference Approximation}. 

\subsection{Forward Finite-Divided Difference Approximation}

Taylor's Theorem can also be stated in terms of a step size, $h<1$, by letting $h=x-x_0$ which implies that $x=x_0+h$. Thus, Taylor's approximation becomes
\begin{equation}
f(x_0+h) = f(x_0) + f'(x_0)h + \frac{f''(x_0)}{2}h^2 + \frac{f'''(\xi_1)}{6}h^3
\end{equation}
where $\xi_1$ lies between $x_0$ and $x_0+h$.

Solving for $f'(x_0)$ in (1) gives
\begin{align*}
f'(x_0)h =& f(x_0+h) - f(x_0) - \frac{f''(x_0)}{2}h^2 - \frac{f'''(\xi_1)}{6}h^3 \\
f'(x_0) =& \frac{f(x_0+h) - f(x_0)}{h}  - \frac{f''(x_0)}{2}h - \frac{f'''(\xi_1)}{6}h^2
\end{align*}
or
\[
\boxed{f'(x_0) \approx \frac{f(x_0+h) - f(x_0)}{h}} \text{ with error } - \frac{f''(x_0)}{2}h - \frac{f'''(\xi_1)}{6}h^2.
\]
\\
{\bf Note on the error term:} The most important part of the error term is the $h$ term with the smallest degree because $h<1$. Also, we generally do not care about the sign or constant coefficients. To denote this we use Big-Oh notation. So, for this approximation we say the error is $O(h)$. In general, for differentiation approximations, $O(h)$ approximations are the worst ones. We now derive another $O(h)$ approximation formula.

\subsection{Backward Finite-Divided Difference Approximation}

Consider yet another form of Taylor, where $x=x_0-h$, thus $-h=x-x_0$ and
\begin{equation}
f(x_0-h) = f(x_0) - f'(x_0)h + \frac{f''(x_0)}{2}h^2 - \frac{f'''(\xi_2)}{6}h^3
\end{equation}
where $\xi_2$ lies between $x_0-h$ and $x_0$.

Solving for $f'(x_0)$ in (2) gives
\begin{align*}
f'(x_0)h =& f(x_0) - f(x_0-h) + \frac{f''(x_0)}{2}h^2 - \frac{f'''(\xi_2)}{6}h^3 \\
f'(x_0) =& \frac{f(x_0) - f(x_0-h)}{h} + \frac{f''(x_0)}{2}h - \frac{f'''(\xi_2)}{6}h^2
\end{align*}
or
\[
\boxed{f'(x_0) \approx \frac{f(x_0) - f(x_0-h)}{h}} \text{ with error } O(h).
\]

As I stated above, this is just another $O(h)$ derivative approximation. If we want something better we need to get rid of the $h$ term in the errors above. We accomplish this by combining multiple formulas together in different ways. The first one we will derive is the {\it Central Finite-Divided Difference Approximation}.

\subsection{Central Difference Approximation}

A more accurate approximation to a first derivative can be obtained by taking a linear combination of two Taylor Theorem approximations. Thus, subtracting (2) from (1) gives 

\begin{align*}
f(x_0+h)-f(x_0-h)=&\left[f(x_0) + f'(x_0)h + \frac{f''(x_0)}{2}h^2 + \frac{f'''(\xi_1)}{6}h^3 \right] \\
 &~~~~~~~~~ -\left[f(x_0) - f'(x_0)h + \frac{f''(x_0)}{2}h^2 - \frac{f'''(\xi_2)}{6}h^3 \right] \\
=& ~2f'(x_0)h+\frac{f'''(\xi_1)}{6}h^3+\frac{f'''(\xi_2)}{6}h^3
\end{align*}
and solving for $f'(x_0)$ gives
\begin{align*}
2f'(x_0)h=&f(x_0+h)-f(x_0-h)-\frac{f'''(\xi_1)+f'''(\xi_2)}{6}h^3 \\
f'(x_0)=&\frac{f(x_0+h)-f(x_0-h)}{2h}-\frac{f'''(\xi_1)+f'''(\xi_2)}{12}h^2
\end{align*}
Thus,
\[
\boxed{f'(x_0) \approx \frac{f(x_0+h) - f(x_0-h)}{2h}} \text{ with error } O(h^2).
\]
This approximation formula is a better method by one degree in the step size as far as potential error.

\subsubsection{ Example 1} Use forward and central difference approximations to estimate the first derivative of
\[
f(x)=-0.1x^4-0.15x^3-0.5x^2-0.25x+1.25
\]
at $x=0.5$ using step size $h=0.5$. Note: the true value is $f'(0.5)=-0.9125$.
\\

\noindent
{\bf Solution:} Using the forward difference formula gives,

\begin{align*}
f'(0.5) &\approx \frac{f(0.5+0.5)-f(0.5)}{0.5} \\
&=\frac{f(1)-f(0.5)}{0.5}\\
&=\frac{0.25-0.975}{0.5}\\
&=-1.45
\end{align*}
The true relative error for this approximation is,
\[
|\varepsilon_t|=\left|\frac{-0.9125+1.45}{-0.9125}\right|=0.589...\approx58.9\%
\]

Using the central difference formula gives,
\begin{align*}
f'(0.5) &\approx \frac{f(0.5+0.5)-f(0.5-0.5)}{2(0.5)} \\
&=\frac{f(1)-f(0)}{1}\\
&=0.25-1.25\\
&=-1
\end{align*}
The true relative error for this approximation is,
\[
|\varepsilon_t|=\left|\frac{-0.9125+1}{-0.9125}\right|=0.09589...\approx9.6\%
\]

As you can see, there is a significant increase in the accuracy of the central ($O(h^2)$) method. The next question is then, can we do even better?

\subsection{High-Accuracy Differentiation Formulas}

Still more accurate approximations to $f'(x_0)$ can be obtained by taking linear combinations of more Taylor Theorem approximations and solving for $f'(x_0)$.  For example, Taylor Theorem approximations for each of
\[
f(x_0-2h),f(x_0-h),f(x_0+2h),f(x_0+h)
\]
can be used to derive an $O(h^4)$ approximation. Consider the following four applications of Taylor's theorem,

\begin{align}
&f(x_0-2h)=f(x_0)-2hf'(x_0)+2h^2f''(x_0)-\frac{8h^3}{6}f'''(x_0)+\frac{16h^4}{24}f^{(4)}(x_0)+O(h^5) \\
&f(x_0-h)=f(x_0)-hf'(x_0)+\frac{h^2}{2}f''(x_0)-\frac{h^3}{6}f'''(x_0)+\frac{h^4}{24}f^{(4)}(x_0)+O(h^5) \\
&f(x_0+h)=f(x_0)+hf'(x_0)+\frac{h^2}{2}f''(x_0)+\frac{h^3}{6}f'''(x_0)+\frac{h^4}{24}f^{(4)}(x_0)+O(h^5) \\
&f(x_0+2h)=f(x_0)+2hf'(x_0)+2h^2f''(x_0)+\frac{8h^3}{6}f'''(x_0)+\frac{16h^4}{24}f^{(4)}(x_0)+O(h^5)
\end{align}

Now, we combine these four equations by taking $(3)-8\times(4)+8\times(5)-(6)$.

\begin{align*}
f(x_0-&2h)-8f(x_0-h)+8f(x_0+2h)-f(x_0+h) \\
&=(1-8+8-1)f(x_0) + (-2+8+8-2)hf'(x_0)+(2-4+4-2)h^2f''(x_0) \\
&~~~~~~~+(-\frac{8}{6}+\frac{8}{6}+\frac{8}{6}-\frac{8}{6})h^3f'''(x_0)+(\frac{16}{24}-\frac{1}{24}+\frac{1}{24}-\frac{16}{24})h^4f^{(4)}(x_0)+O(h^5)\\
&=12hf'(x_0)+O(h^5)
\end{align*}

Rearranging the above equation and solving for $f'(x_0)$ gives,

\[
\boxed{f'(x_0) \approx \frac{1}{12h} \left [f(x_0-2h)-8f(x_0-h)+8f(x_0+h)-f(x_0+2h) \right ]} \text{ with error } O(h^4).
\]

FIGURES 23.1 to 23.3 on pages 656-657 in the text summarize other such formulas.


\subsection{Numerical Differentiation Formulas for Higher Derivatives}

Similar techniques can be applied to find formulas for higher derivatives. For example, we can derive $f''(x_0)$ by doing the following:

Let $n=3$ and use the following Taylor Approximations
\begin{equation*}
f(x_0+h) = f(x_0) + hf'(x_0) + \frac{h^2}{2}f''(x_0) + \frac{h^3}{6}f'''(x_0) + \frac{h^4}{24}f^{(4)}(\xi_1)
\end{equation*}
and
\begin{equation*}
f(x_0-h) = f(x_0) - hf'(x_0) + \frac{h^2}{2}f''(x_0) - \frac{h^3}{6}f'''(x_0) + \frac{h^4}{24}f^{(4)}(\xi_2)
\end{equation*}

Adding these gives:
\begin{equation*}
f(x_0+h)+ f(x_0-h)= 2f(x_0) + h^2f''(x_0) +  \frac{h^4}{24}\left[f^{(4)}(\xi_1)+f^{(4)}(\xi_2)\right]
\end{equation*}
and solving for the desired derivative $f''(x_0)$ gives
\begin{equation*}
f''(x_0)= \frac{f(x_0+h)-2f(x_0)+f(x_0-h)}{h^2} - \frac{h^2}{24}\left[f^{(4)}(\xi_1)+f^{(4)}(\xi_2)\right]
\end{equation*}
That is, 
\begin{equation*}
f''(x_0) \approx \frac{f(x_0+h)-2f(x_0)+f(x_0-h)}{h^2}
\end{equation*}
with a truncation error of $O(h^2)$.



\end{document} 















