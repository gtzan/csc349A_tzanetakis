\documentclass [titlepage,12pt,letter] {article}
\pagestyle{myheadings}

\usepackage{enumerate}
\usepackage{pgf,pgfarrows,pgfnodes,pgfautomata,pgfheaps,pgfshade}
\usepackage{graphicx} 
\usepackage{epsfig}
\usepackage{subfigure}
\usepackage{fancyhdr}
\usepackage{url} 
\usepackage{amsmath}
\usepackage{algorithm} 
\usepackage{algorithmic}
\pagestyle{fancy}

\makeatletter
\renewcommand*\env@matrix[1][*\c@MaxMatrixCols c]{%
  \hskip -\arraycolsep
  \let\@ifnextchar\new@ifnextchar
  \array{#1}}
\makeatother

\DeclareMathOperator*{\argmax}{arg\,max}

\fancyhead{}
\fancyfoot{}
			
\lhead{CSC349A Lecture Notes}
\rhead{Little, Rich}


\setcounter{page}{1}
\cfoot{\thepage}




\begin{document} 


These are the lecture notes for CSC349A Numerical Analysis taught by
Rich Little. They roughly correspond to
the material covered in each lecture in the classroom but the actual
classroom presentation might deviate significantly from them depending
on the flow of the course delivery. They are provide as a reference to
the instructor as well as supporting material for students who miss
the lectures. They are simply notes to support the lecture so the text
is not detailed and they are not thoroughly checked. Use at your own
risk. They are complimentary to the handouts. Many thanks to all the
guidance and materials I received from Dale Olesky who has taught this
course for many years and George Tzanetakis. 


\section{Systems of Linear Equations}


\begin{itemize}
\item{Math background pg. 235-245}
\item{Review if necessary}
\begin{itemize}
\item{matrix notation}
\item{transpose}
\item{matirx operations - addition and multiplication}
\item{inverses, determinants}
\item{triangular matirces}
\end{itemize}
\item{For motivations in Engineering see Chapter 12 Case Studies - pg. 325-336}
\begin{itemize}
\item{Steady-state analysis of systems of reactors}
\item{Analysis of a statically determinate truss}
\item{Currents and voltages in resistor circuits}
\item{Spring-mass systems}
\end{itemize}
\end{itemize}


\section{Naive Gaussian Elimination}


{\bf Notation} for a system of $n$ linear equations in $n$ unknowns:

\[
Ax=b
\]

\noindent
where

\begin{itemize}
\item{$A$ is an $n \times n$ nonsingular matirx, and}
\item{$x$ and $b$ are column vectors with $n$ entries.}
\end{itemize}

That is, given $A$ and $b$, solve for $x$ in

\begin{eqnarray*}
a_{11}x_1 +a_{12}x_2+ \dotsb +a_{1n}x_n &=&b_1 \\
a_{21}x_1 +a_{22}x_2+ \dotsb +a_{2n}x_n &=&b_2 \\
\vdots \\
a_{n1}x_1 +a_{n2}x_2+ \dotsb +a_{nn}x_n &=&b_n
\end{eqnarray*}

\noindent
or

\[
\begin{bmatrix}
    a_{11}       & a_{12} & \dots & a_{1n} \\
    a_{21}       & a_{22} & \dots & a_{2n} \\
    \vdots & \vdots & \ddots & \vdots \\
    a_{n1}       & a_{n2} & \dots & a_{nn}
\end{bmatrix}
\begin{bmatrix}
    x_{1} \\
    x_{2}  \\
    \vdots \\
    x_{n} 
\end{bmatrix}
=
\begin{bmatrix}
    b_{1} \\
    b_{2}  \\
    \vdots \\
    b_{n} 
\end{bmatrix}
\]


You can apply any of the following 3 elementary row operations to $Ax=b$ without
changing the solution.

\begin{enumerate}[i]
  \item multiply any equation $E_i$ by a nonzero constant $\lambda$
  \item replace equation $E_i$ by $E_i + \lambda E_j$
  \item interchange any two equations $E_i$ and $E_j$
\end{enumerate}

{\bf Naive Gaussian Elimination} uses operation ii only to do the following:
\begin{enumerate}
  \item reduce the coefficient matrix $A$ to upper triangular form - {\bf forward elimination}
  \item solve the reduced matrix for $x$ - {\bf back-substitution}
\end{enumerate}

 \subsection{Example 1}
Solve for the following system of linear equations, with $n=3$, using Naive Gaussian Elimination.
\begin{eqnarray*}
x_1+x_2-x_3 &=& -2\\
2x_1-x_2+3x_3 &=& 14 \\
-x_1-2x_2+x_3 &=& 3
\end{eqnarray*}

Here, the augmented matrix $[A|b]$ is

\[
\begin{bmatrix}[rrr|r]
    1 & 1 & -1 & -2 \\
    2 & -1 & 3 & 14 \\
    -1 & -2 & 1 & 3
\end{bmatrix}
\]

{\bf Forward elimination:}

We want $a_{21}=a_{31}=0$, so let $m_{21}=a_{21}/a_{11}=2$ and $m_{31}=a_{31}/a_{11}=-1$. Then,

\[
E_2=E_2-m_{21}E_1
\]
\[
E_3=E_3-m_{31}E_1
\]

so

\begin{align*} 
&a_{21}=a_{21}-m_{21}a_{11}=2-2(1)=0 \\
&a_{22}=a_{22}-m_{21}a_{12}=-1-2(1)=-3 \\
&a_{23}=a_{23}-m_{21}a_{13}=3-2(-1)=5 \\
&b_2=b_2-m_{21}b_1=14-2(-2)=18
\end{align*}
and
\begin{align*} 
&a_{31}=a_{31}-m_{31}a_{11}=-1-(-1)(1)=0 \\
&a_{32}=a_{32}-m_{31}a_{12}=-2-(-1)(1)=-1 \\
&a_{33}=a_{33}-m_{31}a_{13}=1-(-1)(-1)=0 \\
&b_3=b_3-m_{31}b_1=3-(-1)(-2)=1
\end{align*}

Thus, the augmented matrix is now

\[
\begin{bmatrix}[rrr|r]
    1 & 1 & -1 & -2 \\
    0 & -3 & 5 & 18 \\
    0 & -1 & 0 & 1
\end{bmatrix}
\]

Now, let $m_{32}=a_{32}/a_{22}=1/3$ and do $E_3=E_3-m_{32}E_2$.

\begin{align*} 
&a_{31}=a_{31}-m_{32}a_{21}=0-(1/3)(0)=0 \\
&a_{32}=a_{32}-m_{32}a_{22}=-1-(1/3)(-3)=0 \\
&a_{33}=a_{33}-m_{32}a_{23}=0-(1/3)(5)=-5/3 \\
&b_3=b_3-m_{32}b_2=1-(1/3)(18)=-5
\end{align*}

So, finally in upper triangular form the final augmented matrix is

\[
\begin{bmatrix}[rrr|r]
    1 & 1 & -1 & -2 \\
    0 & -3 & 5 & 18 \\
    0 & 0 & -5/3 & -5
\end{bmatrix}
\]

{\bf Back substitution:}
We now solve for $x_1$, $x_2$, and $x_3$ using back substitution.

\begin{align*} 
&-5/3x_3=-5 \Rightarrow x_3=\frac{-5}{-5/3}=3 \\
&-3x_2+5x_3=18 \Rightarrow -3x_2+5(3)=18 \Rightarrow -3x_2=3 \Rightarrow x_2=-1 \\
&x_1+x_2-x_3=-2 \Rightarrow x_1=-2-(-1)+3 \Rightarrow x_1=2
\end{align*}

Therefore,

\[
x =
\begin{bmatrix}[r]
    2 \\
    -1 \\
    3
\end{bmatrix}
\]

{\bf In General:}
After forward elimination we have the augmented matrix

\[
\begin{bmatrix}[cccc|c]
    a_{11}       & a_{12} & \dots & a_{1n} & b_1 \\
    0       & a_{22}^{(1)} & \dots & a_{2n}^{(1)} & b_2^{(1)} \\
    \vdots & \vdots & \ddots & \vdots & \vdots \\
    0       & 0 & \dots & a_{nn}^{(n-1)} & b_n^{(n-1)}
\end{bmatrix}
\]

and then we back-substitute

\[
x_n = \frac{b_n^{(n-1)}}{a_{nn}^{(n-1)}}, x_i = \frac{b_i^{(i-1)} - \sum_{j=i+1}^{n} a_{ij}^{(i-1)}x_j}{a_{ii}^{(i-1)}}
\]


{\bf Notes:}
\begin{enumerate}
\item{When implemented we overwrite $A$ and $b$, thus no need for the superscripts.}
\item{Don't need to calculate, nor store the lower $0$'s.}
\item{The entries $a_{11}, a_{22}^{(1)}, a_{33}^{(2)}, ..., a_{n-1,n-1}^{(n-2)}$ are called the \underline{pivots}. 
They are the denominators in the multipliers - \underline{must be nonzero}}.
\end{enumerate}

\section{Pseudocode for Naive Gaussian Elimination} 


\begin{algorithm}[H]
\begin{algorithmic}[1]
\FOR{$k=1$ to $n-1$}
\FOR{$i=k+1$ to $n$}
\STATE $factor = a_{i,k} / a_{k,k}$
\FOR{$j=k+1$ to $n$}
\STATE $a_{i,j} = a_{i,j} - factor \times a_{k,j}$
\ENDFOR 
\STATE $b_i = b_i - factor \times b_k$
\ENDFOR
\ENDFOR
\end{algorithmic}
\caption{pseudocode for the calculation of forward elimination}
\label{alg:seq}
\end{algorithm}


\begin{algorithm}[H]
\begin{algorithmic}[1]
\STATE $x_n = b_n / a_{n,n}$
\FOR{$i=n-1$ to $1$}
\STATE $sum = b_i$
\FOR{$j=i+1$ to $n$}
\STATE $sum = sum - a_{i,j} \times x_j$
\ENDFOR 
\STATE $x_i = sum / a_{i,i}$ 
\ENDFOR
\end{algorithmic}
\caption{pseudocode for the calculation of back substitution}
\label{alg:seq}
\end{algorithm}


\section{Counting floating point operations} 

The most common measure of the efficiency of algorithms for solving
linear systems $Ax =b$ is a floating-point operation (flop)
count. This is determined as a function of $n$, the order of the matrix
$A$.

The analysis uses the following identities: 
\begin{align*} 
\sum_{i=1}^{m} i = 1 + 2 + 3 + \dots + i = \frac{m(m+1)}{2} \\ 
\sum_{i=1}^{m} i^2 = 1^2 + 2^2 + 3^2 + \dots + i^2 = \frac{m(m+1)(2m+1)}{6}
\end{align*} 

Analysis of the Naive Gaussian Elimination algorithm: count the number
of floating point add/subtracts and multiplies/divides.

The forward elimination iterates over the rows of the matrix to
calculate the factors (or multipliers) and each time there is one less
row to consider. Therefore the number of divides is:
\[
(n-1) + (n-2) + \dots + 1 = \sum_{i=1}^{n-1} i = \frac{(n-1)n}{2} 
\]

For each one of these multipliers all the entries in the corresponding
row (except the zeros) are subtracted and multiplies. Therefore 
the number of subtractions and the number of multiplications are both: 

\[
(n-1)^2 + (n-2)^2 + \dots + 1^2 = \sum_{i=1}^{n-1} i^2 = \frac{(n-1)(n)(2n-1)}{6}
\]

We also have to do the corresponding subtracts and multiplies for $b$ both of which are: 

\[
(n-1) + (n-2) + \dots + (1) = \frac{(n-1)n}{2}
\]

Therefore the forward elimination totals are: 
\begin{align*} 
\mbox{number of mult/div} = \frac{2n^3 + 3n^2 -5n}{6} \\ 
\mbox{number of add/sub} = \frac{2n^3 - 2n}{6}
\end{align*} 

{\bf Back Substitution}
We have one initial division followed by: 
\begin{align*}
 \mbox{number of subtracts} = 1 + 2 + \dots + (n-1) = \frac{(n-1)n}{2} \\ 
\mbox{number of multiplies} = 1 + 2 + \dots + (n-1) = \frac{(n-1)n}{2}
\end{align*} 
\noindent 
with the last number of divides being $n-1$. 

Therefore for back substituion the total is: 
\begin{align*}
\mbox{number of mult/div } = \frac{n^2-n}{2} \\ 
\mbox{number of add/sub } = \frac{n^2-n}{2}
\end{align*}

{\bf Discussion} 
The purpose of the flop count is to give a measure of how computation
time increases as the problem size $n$ increases. For example, it can
help us determine that the forward elimination is much more time
consuming than the back substitution. 

For {\bf Forward elimination} we have
\[
\approx \frac{n^3}{3} + \frac{n^3}{3} = \frac{2n^3}{3} \mbox{flops} 
\] 
\noindent 
The notation $O(n^3)$ is used to indicate the order of increase with problem size without taking into account the constants. 

For {\bf Back substitution} we have 
\[
\frac{n^2 - n}{2} + \frac{n^2-n}{2} + 1 + (n-1) = n^2 
\]

For example, a $2n \times 2n$ linear system requires about 8 times the computation time than an $n \times n$ linear system 
as the cost of the forward elimination step would be 
\[
\frac{2(2n)^3}{3} = 8(\frac{2n^3}{3}).
\]

\section*{Appendix} 

Carl Friedrich Gauss (1777-1855) was an infant prodigy. Gauss enjoyed numerical computation as a child. One day, in order to keep
the class occupied, the teacher has the students add up all the numbers from one to a hundred, with instructions that each should place their slate
on the table in front of them when they are done. Almost immediately Gauss placed his slate on the table. Guass got the answer correct, 5050, by mentally
figuring out that the sum is equal to $100 \times 101/2$. Note that this is the summ identity $\sum_{i=1}^{n} i = 1 + 2 + 3 + \dots + i = \frac{n(n+1)}{2}$.

\end{document}


